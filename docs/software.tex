\chapter{Software}

This chapter explains the software setup of the head and compute nodes.

Software:

\begin{itemize}
	\item \href{https://www.denx.de/wiki/U-Boot}{U-Boot}
	\item \href{https://www.debian.org/}{Debian}
\end{itemize}

\section{Head Node}

As OS a custom Debian image is used and stored on the SD card.
U-Boot, also present on the SD card, is used as bootloader.
The project repository contains a collection of scripts which builds the bootloader and image for you.

All compute nodes are listed in the head node's hosts file.
It's IP address is static.
It runs a DHCP service for the compute nodes.

A service writes the \texttt{authorized\_keys} file for the compute nodes.

At the point of writing we did not manage to get the M.2 SSD drive to work.

\section{Compute Node}

The compute nodes use U-Boot as bootloader on their SD cards.
However, the OS image is located in the head node's eMMC memory and loaded via TFTP.
The head node's IP address and kernel image path are hardcoded in U-Boot.

All compute nodes share the same OS image, which is mounted read-only.
OverlayFS is used to enable writing.
All changes are stored in RAM, hence they are lost on reboot.

IPs are determined via DHCP.

\subsection{INA219}

The power measurement values can be read from somewhere in \texttt{/sys/class/hwmon}.
